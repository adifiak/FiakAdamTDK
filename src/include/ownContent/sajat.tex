\chapter{Saját kontribúció}
Ebben a fejezetben kerülnek bemutatásra a téma kapcsán létrejött saját eredmények. Először a megoldás lehetséges módjai kerülnek bemutatásra az előnyeikkel és hátrányaikkal együtt.
A második alfejezet a rendszerek SysML v2-ben történő modellezéséhez létrehozott módszertanokat mutatja be,amelybe beleépülnek a szimulációs technológiákhoz létrehozott módszerek.
Végül egy konkrét esettanulmányon keresztül bemutatásra kerülnek a leírt módszerek.

\section{Lehetséges megközelítések}
    A modellalapú rendszertervezési és a szimulációs technológiák integrálása módosításokat követel mind a két oldal modelljeiben.
    Ezek a módosítások és tervezési módszerek a két terület közötti kapcsolat, egy kölcsönös megfeleltetés miatt elengedhetetlenek.
    Azonban több különböző lehetőségünk van, ennek a megvalósítására.
    Ez a fejezet ezeket a megközelítéseket veszi sorra és hasonlítja őket össze.

    \subsection{Szimulációs fogalmak bevezetése a rendszermodellbe}
        A rendszermodell tekintetében a leglátványosabbak ezek a változások, mivel itt ténylegesen meg kell jelenni a másik szakterületnek, hogy az beolvadhasson a fejlesztési folyamatba.
        A szimulációs fogalmak bevezetésére két módszert dolgoztunk ki, amelyeket részletesen tárgyalunk majd.

        \subsubsection{Annotáció alapú leképezés}
        Az új fogalmak és tulajdonságok bevezetésére a legkézenfekvőbb megoldás a SysML v2 nyújtotta lehetőség különböző metaadatok bevezetésére.
        A megoldás lényege, hogy létrehozunk egy importálható fájlt amely minden fontos fogalomhoz tartalmaz egy-egy annotációt, amivel az adott eleme bevezethető.
        Ezen kívül lehetőség van szabályokat megadni, amelyek segítenek megakadályozni a hibás használatot, például nem jelölhetjük külön szimulációs egységnek (Például FMU-k.) a követelményeket vagy a köztük lévő logikai kapcsolatot.
        Ezzel a megoldással, a fogalmak bevezetése abból áll, hogy minden elemet ellátunk a megfelelő annotációval és ezek alapján a leképezést létrehozó vagy felhasználó programok ezek segítségével tudják értelmezni, bejárni a rendszermodellt szimulációs szemszögből nézve.

        \subsubsection{Szimulációs modellvetület bevezetése}
        A másik lehetőség, ha bevezetünk egy konkrét területet a modellben -egyfajta nézetet- amely leírja a szimulációs modellt.
        A rendszermodellezésben alapvetően jelen van ez a szemlélet, a rendszerek tervezésénél egy logikai és egy fizikai modellel dolgoznak a rendszermérnökök.
        Ennél a megközelítésnél egy ezekhez hasonló, harmadik vetületet hozunk be a rendszermodellbe és a rendszermodell elemeit explicite összekötjük az azokat reprezentáló szimulációs elemek modelljével.
        Ezzel a megoldással a szimulációkhoz kapcsolódó fogalmak, csak a vonatkozó nézetben jelennek meg.
        Magának a nézetnek a megvalósításához célszerű készíteni egy nyelvi kiterjesztést.
        Ez a SysML v2 adta lehetőség arra, hogy szakterület specifikus típusokat hozhassunk létre.
        Ennek az eszköznek a használata célszerűbb ennél a megközelítésnél, mert itt egy konkrét szakterület szerinti modellt alkotunk.
        Nincs szükség az annotációk flexibilitására, cserébe a használata egyszerűbb, jobban kikényszeríthető a megfelelő struktúra alkalmazása, mint ha csak különböző annotációkat adnánk az egyes elemekhez.

        \subsubsection{Lehetőségek összehasonlítása}
        Az annotáció alapú megközelítés előnye könnyen látható. Az, hogy a meglévő modell elemeit felcímkézzük, lehetővé téve a leképezést, azt jelenti, hogy minimális többletmunkára van szükség az eddigiekhez képest.
        A kevesebb nézet szerinti modellezés kevesebb munkát igényel, így a fejlesztés gyorsabban haladhat.
        
        A módszernek azonban van három komoly hátránya is, amelyek a második módszer mellett könnyen megoldhatóak.

        Ha a teljes modellben jelen vannak az új szakterület fogalmai, az azt jelenti, hogy mindenkinek tisztában kell lenni azokkal, hogy dolgozni tudjon.
        Ez azt jelenti, hogy minden rendszermérnököt külön szimulációs technológiákra vonatkozó képzésben kell részesíteni.
        Ráadásul az extra képzésben részesített szakemberek nyilvánvalóan nem tudják olyan hatékonyan végezni ezt a munkát, mint azok, akik külön erre a területre specializálódtak.

        Az annotációs módszer második hátránya, hogy bármilyen modellstruktúrát megenged, nem kényszeríti ki egy jól kidolgozott módszertan követését.
        Egy általános modellstruktúra leképezéséhez a feladatot ellátó szoftvereknek is jóval komplexebbnek kell lenniük, mintha egy előre ismert struktúra mentén dolgoznának.

        A harmadik probléma, hogy az elemeken elhelyezett annotációk egyféle képpen címkézik fel a modellt, azaz egyetlen szimulációs modellt írnak le.
        A szimulációk során azonban rendszerint számos különböző modellt használnak, az aktuális igényeknek megfelelően, ezek között szükséges hogy lehessen váltani, a komponensek különböző modelljeiből több összeállítást készíteni.
        Ehhez vegyünk például egy robotot és annak tápegységét.
        Amikor a tápegységet fejlesztjük, fontos, hogy a benne zajló jelenségeket minél pontosabban modellezzük. például egy kapcsoló üzemű tápegységnél a benne használt tranzisztorok pontos viselkedése elengedhetetlen a hatékonyság és a kimenő vezetékeken megjelenő zaj optimalizálásához, a követelmények teljesülésének vizsgálatához.
        Azonban amikor a robotunk vezérlését teszteljük, akkor rendszerint érdektelen számunkra, hogy tökéletesen pontosan ismerjük a fenti paramétereket, az a fontos, hogy az adott mozgások okozta terhelést képes-e kielégíteni a rendszer, illetve körülbelül mekkora a fogyasztása.
        Ezt a problémát úgy oldják meg, hogy több, különböző részletességű modellt készítenek, az egyes feladatokhoz.
        Persze használható minden feladatra a legrészletesebb modell, azonban az extra részletek sokszor nem jelentenek lényeges különbséget, míg a szimuláció számításigényét jelentősen megemelhetik.

        A szimulációs modellhez tartozó nézet bevezetése megoldja a fenti problémákat.

        Azzal, hogy egy konkrét modellrészletre korlátozzuk a speciális tudást igénylő munkafolyamatokat, lehetőség nyílik egyetlen, specializált szakemberekből álló csoportra bízni ezt a feladatot.
        A többi mérnöknek nem kell további képzés, a specializált szakemberek pedig jobb minőségű munkát végezhetnek a terület behatóbb ismeretének köszönhetően.

        Az előre ismert struktúra és a domain-specifikus modellezési eszközök bevezetése megkönnyíti a modelleket és az eszköztámogatást készítők munkáját is.

        Azzal, hogy a különböző szimulációs funkciókat nem kötjük hozzá direkt módon a rendszermodell elemeihez, lehetőség nyílik, hogy a szimulációs nézetben több szimulációs modellt is kapcsoljunk egy rendszerkomponenshez.
        Ez lehetővé teszi az eltérő részletességű modellek bevezetését, valamint a HiL és SiL technológiákhoz kapcsolódó szimulációs modellek is könnyen kezelhetőek egy extra verzióként.

        A fentiek alapján látható, hogy az új nézet bevezetése jobban illeszkedik a technológiák alkalmazásához, jobb minőségű modelleket tesz lehetővé és a speciális szaktudás igényének csökkentésével az ilyen módszerek egyszerűbben és olcsóbban vezethetőek be a gyakorlatban.
        Ezek alapján a további munka során erre a megközelítésre összpontosítunk.

    \subsection{Szimulációs modell architektúrák}
        A szimulációs modellek esetében a változások azért szükségesek, hogy a rendszermodell megfelelően leképezhesse a szimulációs modelleket.
        Itt inkább tervezési konvenciókról, megkötésekről van szó, mivel egy teljesen szabadon készült modell leképezése jóval bonyolultabb feladat, mintha valamilyen előre ismert sémát használunk.

        \subsubsection{Leképezés a Modelica teljes eszközkészletével}
            A Modelica nyelvet kifejezetten fizikai rendszerek leírására és szimulációjára fejlesztették ki.
            Ennek köszönhető gazdag eszközkészlete, a különböző jelenségek leírására.
            Az első megközelítés, hogy ezeknek az eszközöknek mindegyikét szabadon használjuk a rendszer leírásához, mivel a minél pontosabb modellek érdekében hozták létre ezeket a nyelvi elemeket.
            Ilyen eszköz a külső változók használata, amelyek a környezeti hatások, mint például a gravitáció megadását teszik lehetővé. Ez azért fontos mert például egy bolygó modelljében, csak azt tüntetjük fel, hogy van ilyen érték és ez hogyan befolyásolja ezt a modellelemet, azonban a számítását a környezetre bízza, így könnyebben újrahasználható és logikusabban strukturált modelleket kapunk.
            A másik fontos lehetőség az úgynevezett akauzális csatlakozók használata. Ezek irányítatlan kapcsolatokat jelölnek, ami azt jelölő, hogy a kapcsolat két vége valamilyen szempontból egyenlő egymással. Jó példa erre az áramköri elemek beépített modellkönyvtára. A kapcsolat itt a kapcsolási rajzok vezetékeivel analóg módon viselkedik, az áram mindkét irányba folyhat az adott kapcsolaton és mindkét oldalán található komponensek befolyásolják ennek mértékét.
            Ez szintén olyan jelenség, amely a világban igen gyakran megfigyelhető -gyakorlatilag minden fizikai mennyiségnél-, és rengeteg segítséget nyújt, amikor ilyen viselkedést kell modellezni.
            Könnyen belátható, hogy ezeknek a lehetőségeknek a korlátlan kihasználása mennyivel egyszerűbbé teszi a szimulációs modellek megalkotóinak munkáját.

        \subsubsection{FMU kompatibilis leképezés}
            Az FMI szabványt azért hozták létre, hogy az egyes modelleket könnyen át lehessen vinni egyik eszközből a másikba, illetve megoszthatóak, újrahasználhatóak legyenek az önálló logikai egységek.
            Ez egy olyan képesség, amit célszerű kihasználni az összetett rendszerek tervezése során, ez azonban megszorításokat kényszerít a modellekre.
            Az eszközfüggetlenség ugyanis azt jelenti, hogy a szabvány csak a létező legalapvetőbb adatkapcsolatokat teheti lehetővé, amelyek minden eszközben elérhetőek lesznek.
            Ez a gyakorlatban azt jelenti, hogy kizárólag egész és racionális számok valamint logikai értékek szerepelhetnek a csomag ki és beviteli csatlakozóin és ezek a kapcsolatok kizárólag egyirányúak lehetnek.
            Az ilyen megkötések mellett modellezett komponensek alakíthatóak át FMU-vá, hogy hordozhatóvá tegyék őket.
            Az FMI szempontjából csak a komponens és a környezetének kapcsolata számít, tehát csak azoknak a jeleknek kell megfelelni ezeknek a kritériumoknak, amelyek áthaladnak a komponens határain.

        \subsubsection{Lehetőségek összehasonlítása}
            A fentiek alapján látható, hogy ebben a kérdésben nincsen olyan határozott válasz, mint a rendszermodellek esetében.
            Ahhoz, hogy a modellek minél jobban legyenek rendezve és minél egyszerűbben tegyék lehetővé a rendszerek leírását, használnunk kell az erre szolgáló eszköztárat.
            Ezek az eszközök azonban kizárják a modellek hordozhatóságát, amely elengedhetetlen komplex -több szakterületet érintő- rendszerek esetében, vagy több független szereplő együttműködése esetén.
            Az első lépés ennek a problémának a feloldására, hogy jól definiáljuk, mely komponenseket szeretnénk FMU-vá alakítani, mivel a modellezési eszközök csak ezek határain kerülnek megszorításra.
            Ha továbbra is gondot jelentenek a megkötések, akkor célszerű két részre bontani az adott komponenst -ha ez lehetséges-.
            Az első rész adja az FMU-t, amely tartalmaz minden olyan logikát, amelyet nem szeretnénk megosztani, és elfogadható nehézséggel kommunikál a komponens modelljének maradék részével a megszorítások mellett is.
            A második rész a minimális modellt tartalmazza, amely a megszorított kommunikációs módokat átalakítja egy eszköz specifikus, de könnyen használható interfésszé.
            Fontos, hogy ezek az adapterelemek minél egyszerűbbek legyenek, mivel ezeket minden támogatni kívánt eszközben külön el kell készíteni, valamint ezekre nem terjed ki a források védelme, mivel nem alakíthatóak FMU-vá.
            Hasonló technikával próbálták már feloldani az akauzális kapcsolatok hiányát //TODO hivatkozás a Japán kutatókra, de a módszer még nem elég kiforrott és az alkalmazása csak automatizáció után lehet valós alternatíva a gyakorlati alkalmazásokban.
            Ezen felül, felmerült már az FMI szabvány kiterjesztése is az akauzális kapcsolatok támogatására, azonban ez még csak elméleti lehetőségként létezik, így szabványos megoldás egyenlőre nem várható.

\section{Modellezési módszertan}

\section{Kvadkopter esettanulmány}