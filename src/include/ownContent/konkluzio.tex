\chapter{Konklúzió és további munka}

\section{Összefoglalás}
A dolgozatban leírt munka során először összegyűjtöttem a modellalapú rendszertervezés és a szimulációs technológiák külön-külön illetve együttes alkalmazásának előnyeit.
Ezután Megvizsgáltam egy, a fenti technikákat integráló SysML v2 alapú tervezési módszertan által teljesítendő követelményeket, javaslatot tettem több különböző implementációs irányra, majd kiértékeltem ezeket.
Az eredményeket felhasználva kidolgoztam és egy esettanulmányon keresztül demonstráltam egy új, a fentieknek megfelelő tervezési módszertant, ami számos az integrációból fakadó előnnyel jár, valamint alapot biztosít továbbiak elérésére, ha a hozzájuk szükséges fejlesztőeszközök elkészülnek.
A fejlesztés során feltárásra került, hogy a javasolt előnyök némelyike külön eszközöket igényel, amelyek csak a fejlesztési folyamatban igényelnek helyet a használatukba, magától a módszertantól függetlenek.
Ezek esetében megvizsgáltam, hogy használhatóak-e ezek a technikák, melynek során nem találtam kizáró okot a módszerek együttes használatára.

\section{További kutatási és fejlesztési lehetőségek}
A kódgenerálás kapcsán felmerült, hogy ez az elem elengedhetetlen ahhoz, hogy a módszertan megfeleljen eredeti rendeltetésének, és bemutatásra került a fejlesztés alatt álló SysPhS eszköz is, amely alkalmazásával ez a feladat megoldhat gyorsabban megoldható, mint saját eszköz fejlesztésével.
Ezek alapján konzulensemmel fel is vettük a kapcsolatot az eszközt fejlesztő egyik csapattal és megkezdtük az együttműködést a két eszköz összekapcsolása érdekében.
Jelen dolgozat elsődlegesen a validációs technológiák közül a szimulációk integrálásával foglalkozott, mert ezek igényelnek külön környezetekben készített és futtatott modelleket, azonban a jövőben szeretnék több más V\&V technikát is integrálni a módszertanba és ezek függvényében bővíteni a bemutatott esettanulmányt.
Az esettanulmány kidolgozása folyamán a módszertanban azonosítottam több általános mintát, melyeknek egy SysML v2 nyelvi kiterjesztésben való összegyűjtése tovább egyszerűsítené azok használatát, amely előnyös lenne a módszer elterjesztésének esetén. Természetesen a nyelvi kiterjesztés elkészültével újra frissíteném az esettanulmányt, hogy használható legyen a teljesen kiforrott módszertan bemutatására, valamint referenciamodellnek.