\chapter{Háttérismeretek}

\section{Motiváció}

    \subsection{Modell alapú rendszertervezés előnyei}

        \subsubsection{Nyomon-követhetőség}

        \subsubsection{Modellek konzisztenciája}

        \subsubsection{Validáció és Verifikáció (V\&V) minden fejlesztési fázisban}

        \subsubsection{Modellek újrahasználhatósága}

    \subsection{Szimulációs technológiák előnyei}

        \subsubsection{Megismételhetőség és összehasonlíthatóság}

        \subsubsection{Gazdaságos tesztelés}

        \subsubsection{Szélsőséges esetek tesztelhetősége}

        \subsubsection{Hardware és Software in the Loop tesztek (HiL/SiL)}


\section{Jelenlegi eszközök}

    \subsection{Modellezési nyelvek}

        \subsubsection{SysML}

        \subsubsection{SysML v2}

        \subsubsection{Modelica}

    \subsection{Modellek hordozása és szimulálása}

        \subsubsection{Functional Mock-up Interface (FMI)}

        \subsubsection{System Structure and Parameterization (SSP)}

\section{Alapvető problémák}

    \subsection{Tervezési módszertanok hiánya}

    \subsection{Modellek megfeleltetésének hiánya}

    \subsection{Izolált V\&V}